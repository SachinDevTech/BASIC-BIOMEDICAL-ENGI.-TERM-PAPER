\documentclass[12pts]{article}
\title{\textbf{NATIONAL INSTITUTE OF TECHNOLOGY, RAIPUR(C.G)}}
\usepackage{graphicx}
\graphicspath{{Images/}}


\author{\underline{Submitted By: SACHIN KUMAR}\\sachin.1107sk@gmail.com\\Roll No: 21111047(A047)\\ \underline{Submitted to: Dr. Saurabh Gupta Sir}}
\date{March 16, 2022}
\begin{document}
\maketitle

\begin{figure}[h]
\centering
\includegraphics[scale=0.69]{nit.jpg}
\caption{National Institute of Technology, Raipur}
\end{figure}
\centering
\textbf{TERM PAPER ON \underline{"TRAETING CARDIAC DISEASE WITH CATHETER-BASED TISSUE HEATING"}}
\centering
UNDER THE SUPERVISION OF DR. SAURABH GUPTA SIR\\

\clearpage

\section*{Treating Cardiac Disease With Catheter-Based Tissue Heating}
 

\raggedright \section*{\underline{ABSTRACT:}} 
\begin{Large}

Microwave ablation involves delivering electromagnetic energy to a precise point in a coronary artery via a catheter for selective heating of a particular atherosclerotic lesion. Controlling the power provided, pulse duration, and frequency might result in beneficial temperature profiles. A microwave source, a catheter/transmission line, and an antenna at the distal end of the catheter are the main components of a microwave ablation equipment. The antenna would focus the radiated beam so that the majority of the microwave energy was deposited within the atherosclerotic lesion that was being targeted.
\end{Large}

\section*{INTODUCTION}
\begin{flushleft}
\begin{large}
For decades, scientists have been using electromagnetic and sonic energy to serve medicine. But, aside from electro surgery, their efforts have focused on diagnostic imaging of internal body structures—particularly in the case of x-ray, MRI, and ultrasound systems. Lately, however, researchers have begun to see acoustic and electromagnetic waves in a whole new light, turning their attention to therapeutic—rather than diagnostic—applications. Current research is exploiting the ability of radio-frequency (RF) and microwaves to generate heat, essentially by exciting molecules. This heat is used predominantly to ablate cells. Of the two technologies, RF was the first to be used in a marketable device.
\end{large}
\end{flushleft}

\section*{\underline{CARDIOVASCULAR DISEASE(CVD)}}
\begin{large}
The term "cardiovascular disease" or Heart disease or Cardia Disease refers to any ailment that affects the heart or blood vessels.
It's frequently linked to fatty deposits in the arteries (atherosclerosis) and an elevated risk of blood clots.
It has also been linked to artery damage in organs like the brain, heart, kidneys, and eyes.
CVD is one of the primary causes of death and disability in the United Kingdom, yet it may often be avoided by maintaining a healthy lifestyle.

\begin{figure}[h]
\centering
\includegraphics[scale=.7]{cvd.jpg}
\caption{A glimpse of Heart Diseases}
\end{figure}

\subsubsection*{\underline{Types of CVD:}}
\begin{large}
There are many different types of CVD. Four of the main types are described below.
\end{large}
\begin{itemize}
\item Coronary heart disease
\item Strokes and TIAs
\item Peripheral arterial disease
\item Aortic disease
\end{itemize}

\subsubsection*{\underline{Preventing CVD}}
\begin{large}
CVD risk can be reduced by leading a healthy lifestyle. If you already have CVD, maintaining as healthy as possible will help you avoid worsening your condition.
\end{large}
\begin{enumerate}
\item Stop smoking
\item Have a balanced diet
\item Exercise regularly
\item Maintain a healthy weight
\item Cut down on alcohol
\item Medication
\end{enumerate}
\end{large}

\raggedright \subsection*{\underline{MICROWAVE vs RADIOACTIVE RADIATION}}
\begin{large}
There is a significant difference between these two types of radiation. Microwaves used in microwave ovens, like microwaves used in radar equipment and telephone, television, and radio communication, are in the non-ionizing region of electromagnetic radiation, as shown in the frequency spectrum on the right. Ionizing radiation is not the same as non-ionizing radiation. The frequency of ionising radiation is really high (millions of trillions of cycles per second). As a result, it is highly forceful and penetrating. Ionizing radiation can harm living tissue cells even at modest doses. In fact, the energy and intensity of these hazardous rays is sufficient to alter (ionise) the molecular structure of matter. Ionizing radiation can cause genetic alterations in high enough doses.The ionising range of frequencies includes X-rays, gamma rays, and cosmic rays, as illustrated on the frequency spectrum. The type of radiation we associate with radioactive chemicals like uranium and radium, as well as the fallout from atomic and thermonuclear explosions, is known as ionising radiation. Non-ionizing radiation is not the same as ionising radiation. It does not have the same harmful and cumulative effects as ionising radiation because of the lower frequencies and lower intensity. Microwave radiation (at 2450 MHz) is non-ionizing, and when strong enough, it causes matter molecules to vibrate, generating friction, which creates the heat that cooks the meal.
\end{large}


\raggedright  \subsection*{\underline{BALLOON ANGIOPLASTY}}

\begin{large}
\begin{flushleft}
A catheter with a tiny balloon is carefully inserted into the artery and inflated to expand the aperture and boost blood flow to the heart. During the procedure, a stent is frequently used to maintain the artery open after the balloon has been deflated and removed.

\begin{figure}[h]
\centering
\includegraphics[scale=0.69]{angio.jpg}
\caption{Angioplasty Diagram}
\end{figure}

If the blockage is minor, inflating the balloon multiple times may be enough to solve the problem. The plaque will compact against the wall, expanding the path and allowing blood to flow freely.
A stent, a tubular device, is commonly inserted into the artery at this point. Inside the artery, it will operate as a scaffold, keeping the blood vessel open.
\end{flushleft}
\end{large}


\section*{\underline{MICROWAVE CARDIAC ABLATION}}

\begin{flushleft}
\begin{large}
The therapy of aberrant heart rhythm, or cardiacarrhythmia, is another application of catheter-based microwave heating. Anomaly electrical activity in some parts of the heart causes this life-threatening condition. Although medicines can be used to slow down an abnormally fast heart rate, removing or destroying a piece of this tissue manually is more successful in treating arrhythmias. Selective catheter-fed ablation, or excessive tissue heating, kills the area of the heart that is causing the abnormal electrical activity.

\begin{figure}[h]
\centering
\includegraphics[scale=0.19]{mca.jpg}
\caption{Microwave cardiac ablation}
\end{figure}

The annual number of deaths from Cardiovascular disease (CVD) in India is projected to rise from 2.26 million (1990) to 4.77 million (2020). Coronary heart disease prevalence rates in India have been estimated over the past several decades and have ranged from (1.6 to 7.4) percentage in rural populations and from (1 to 13.2) percentage in urban populations.
In other words we can say, Antenna applicators for microwave catheter ablation (MCA) have been employed in cardiac ablation experiments. Monopolar antennas and helical coil antennas are the two types of applicators available. The normal mode is emitted by both varieties, with waves travelling perpendicular to the helix's axis. Furthermore, monopole antennas are usually half the wavelength of the tissue and produce a well-defined football-shaped heating pattern along their axis.
\end{large}
\end{flushleft}
\section*{\underline{MABA APPLICATOR}}
\begin{large}
A helix positioned inside the balloon and designed to radiate a rapidly oscillating radially polarised field is a viable option. The relative positions of this helical antenna as part of an angioplasty balloon in the blood vessel are shown in Figure 3. A modal study based on the sheath helix model was used to investigate this helical antenna analytically. While the sheath helix model does not incorporate wave launching or termination, it can help with pitch angle selection and overall concept viability. A mode filter, made up of circumferential tiny metallic traces on the expanding balloon, is utilised to ensure that the electric fields radiated into the plaque are radially orientated as closely as possible.The dominant circumferentially oriented electric field, which deposits power adversely in the healthy arterial wall, is eliminated by this filter.
\end{large}

\section*{Electronic devices that helps in minor problem}
\begin{large}
HeartFlow is a novel imaging tool that allows doctors to more readily evaluate if a patient requires an invasive procedure to open blocked arteries or if a non-invasive treatment, such as medication, will suffice.

\begin{figure}[h]
\centering
\includegraphics[scale=0.15]{cardia.png}
\caption{Devices names that are helpful for minor problems}
\end{figure}

\end{large}




\section*{\underline{CONCLUSION}}
\begin{large}
There have been two applications of microwave internal biological heating discussed. An antenna applicator is fed via a coaxial cable that passes through a catheter in both MABA and MCA. To achieve certain power deposition patterns, the antenna designs take advantage of microwave polarization and phase effects. The substantial variations in the dielectric properties of HWC and LWC tissue are used by MABA with a helix and mode filter balloon to preferentially heat and weld plaque while protecting healthy artery walls. The wide aperture MCA generates a deep big ablation volume in sick cardiac tissue by using an unfurlable spiral antenna within a balloon. A number of in-vitro and in-vivo tests have been used to validate theoretical investigations.
\end{large}

\section*{Reference}



\begin{figure}[h]
\centering
\includegraphics[scale=1.5]{refer.jpg}
\caption{Not clickable links}
\end{figure}




\end{document}